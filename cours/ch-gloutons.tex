\chapter{Algorithmes gloutons}
\section{Une manière de procéder...}

\begin{definition}[ : algorithme glouton]
Un algorithme est dit \textit{glouton} lorsque
\begin{itemize}
	\item 	il procède étape par étape, avec une boucle ;
	\item 	à chaque itération il essaye d'\textit{optimiser} une grandeur (maximiser ou minimiser) en faisant un \textit{choix} ;
	\item 	les choix faits sont \textit{définitifs} : ils ne sont jamais remis en questions lors des itérations suivantes.	
\end{itemize}
\end{definition}

\begin{exemple}[ : rendu de monnaie]
Lorsqu'on rend la monnaie en euros et qu'on veut rendre le moins de pièces (ou billets) possibles, on
\begin{itemize}
\item 	procède pièce par pièce ;
\item  	choisit la pièce dont la valeur est la plus grande possible tout en restant inférieure ou égale au montant qu'il reste à rendre ;
\item 	on continue ainsi jusqu'à ce qu'il ne reste plus rien à rendre, sans jamais reprendre une pièce rendue auparavant.
\end{itemize}
Cette méthode est gloutonne et elle permet toujours de rendre la monnaie avec le moins de pièces possible (en tout cas lorsque le système monétaire est l'euro).
\end{exemple}
\section{Qui n'est pas toujours optimale}


Considérons un robot placé en A, qui veut monter le plus haut possible.
S'il applique la méthode gloutonne suivante :\medskip\par
\floatpictureright{0.33}{img/glouton}{
\begin{itemize}
\item à chaque seconde, tant que possible ;
\item regarder à droite ou à gauche sur une petite distance ;
\item aller dans la direction ou la pente est la plus forte.
\end{itemize}
}\medskip\par
Alors il se retrouvera en m, et pas en M.

\begin{aretenir}
\begin{itemize}
\item  un algorithme glouton ne fournit pas toujours une solution optimale ;
\item pour s'assurer qu'il fournit une démonstration optimale, il faut le \textit{démontrer}.
\end{itemize}
\end{aretenir}

\section{Des exemples optimaux}

\begin{itemize}
	\item le rendu de pièces en euros ;
	\item l'écriture d'un entier naturel en binaire par la méthode des soustractions (qui correspond à un rendu de pièces qui ont des valeurs de $2^n$) ;
	\item l'algorithme dit « des conférenciers » ;	
\end{itemize}
Nous en verrons d'autres en Terminale.

\section{Des exemples non optimaux}

\begin{itemize}
	\item le problème du robot exposé précédemment ;
	\item les méthodes gloutonnes pour résoudre le problème du « sac à dos ».
\end{itemize}

